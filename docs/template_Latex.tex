\documentclass[a4paper, 12pt]{article}
% Packages
\usepackage[margin=1.1in]{geometry} % Layout
\usepackage{caption}
\usepackage[hidelinks]{hyperref}
\usepackage{graphicx}
\usepackage{amsmath}

\bibliography{source} % add sources to source.bib file
\usepackage[backend=bibtex,style=verbose-trad2]{biblatex} % or ieee
% https://www.latex-tutorial.com/tutorials/bibtex/

\usepackage{titling} % Damit man  theauthor nutzen kann

%Bild
\graphicspath{{img/}}
\usepackage{float} % Allows putting an [H] (wenn gross H dann bild genau dort)

\usepackage{fancyhdr} % Header & Footer

% Für Deutsch
\usepackage[utf8]{inputenc}
\usepackage[T1]{fontenc}
\usepackage[ngerman]{babel}

\usepackage{siunitx} % Required for alignment
\sisetup{
	round-mode          = places, % Rounds numbers
	round-precision     = 2, % to 2 places
}
\usepackage{booktabs} % For prettier tables 

% Document information
\title{PSIT 4 - Markdoc}
\author{Julian Visser}

% Header & Footer
\pagestyle{fancy}
\fancyhf{}
\lhead{PSIT4}
\chead{Markdoc}
\rhead{\theauthor}
\cfoot{\thepage}

\begin{document}
	
	\pagenumbering{gobble}
	\begin{titlepage}
		\maketitle
	\end{titlepage}


	%Index
	\newpage
	\pagenumbering{roman}
	\tableofcontents
	
	% New page
	\newpage
	\pagenumbering{arabic}
	
	\section{Überschrift 1}
	% Start writing here
	
	%No Praktisch zum footnotes mache zum zitiere
	%\autocite[1]{DUMMY:1}
	
	% New page
	\newpage
	
	\section{Anhang}
	
	\listoffigures
	\listoftables
	\printbibliography  % Uses source.bib to make ref table
	
\end{document}
